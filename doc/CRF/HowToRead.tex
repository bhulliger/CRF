% How to read this document - brods1

\chapter{How to read this documents}

\subsection*{Introduction}
Our project target for the bachelor thesis was to build a framework which allows to render scenes on multiple resources. Consequently, this requirement led to Equalizer. Moreover, it was demanded to build the \gls{crf} on a common scene graph library like \gls{osg}. We soon found synergies with some other students of the University of Siegen who were already working on a similar task. As a matter of fact, they shared their achievements which helped us to get started with the \gls{crf}. Later on, we found out that that not just we, but a growing amount of other Equalizer users would be interested in an integration of \gls{osg} in Equalizer. For that reason we separated project specific documentation, e. g. the project management, from purely framework concerning documentation. Our intention is to share the latter with the Equalizer community. Hence, other users can use the \gls{crf} or even continue development. Because Equalizer users are spread all over the world, we wrote the documentation in English rather than German.

For that reason, keep in mind that the structure of this document differs from an ordinary bachelor thesis report. It is split up into four documents which can be read and distributed independently. In the following, we provide a short explanation how you should read these reports and what target group we had in mind for each document.\\
\\
We provide the following documents:
\begin{itemize}
	\item Technical Report
	\item Thesis Report
	\item User Manual
	\item Functional Specification
\end{itemize}

\section*{Thesis Report}

\subsection*{What it is}
This document focuses on our bachelor thesis. It shows how we approached our project goals, what we have done and what still needs to be done. It delivers an impression of what technologies we used and how we have chosen them. 

\subsection*{Who should read it}
This document is basically meant to be read by our supervisors Prof. U. K\"unzler and M. Luggen, and our expert H. van der Kleij.

\section*{Technical Report}

\subsection*{What it is}
The technical report describes our achievements concerning the integration of \gls{osg} into Equalizer. It contains a detailed description on the implementation and advises other programmers how to make use of it. As well it provides information on the Equalizer setup that we have worked out.

\subsection*{Who should read it}
As mentioned, there is a growing community of Equalizer users who want to take advantage of \gls{osg}. To enable interested parties to use our framework, we want to share this document as technical reference. On this account, we kept the focus in this part on the technical facts rather than organisational information or the progress during the project.

\section*{Developer Manual}

\subsection*{What it is}
On the one hand, the user manual shows \gls{cave} specific information and advices which we gathered during the bachelor thesis. The document provides a well documented guide on how to set up the required environment for our framework.

On the other hand, we want to introduce a simple \gls{osg} demo application which uses the \gls{crf}. Especially, we want to point out how to use the interface to our framework.

\subsection*{Who should read it}
The developer manual is basically meant for the \gls{bfhti} \gls{cpvr} research group or anybody interested in \gls{crf} framework.

\section*{Functional Specification}

\subsection*{What it is}
In the functional specification we specified the initial environment and evaluated the technologies we want to use. Moreover, we created use cases and defined the requirements concerning the \gls{crf}. This document helped to set limits for our bachelor thesis.

\subsection*{Who should read it}
The functional specification is given for the sake of completeness. We refer to it in the thesis report.
