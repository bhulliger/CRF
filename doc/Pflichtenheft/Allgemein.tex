\chapter{Allgemeines}
Das Projekt \textit{CAVE Rendering Framework} ist eine Bachelor Thesis von drei Studenten der Berner Fachhochschule f\"ur Technik und Informatik (TI-BFH) in Biel. Die Arbeit wird in der Vertiefungsrichtung \textit{Computer Perception \& Virtual Reality} (CPVR) erstellt.
Ziel ist es, im Zeitraum Februar bis Juni 2009 selbstst\"andig ein Projekt zu erarbeiten, inklusive der ben\"otigten Projektplanung und Dokumentation.
Betreut wird das Projekt von einem Dozenten der Forschungsgruppe CPVR - Prof. U. K\"unzler, und seinem Assistenten Michael Luggen. Nebst der Betreuung ist ein Experte in das Projekt involviert - Herr Han Van der Kleij. 

\section{Zweck des Dokuments}
Das vorliegende Pflichtenheft ist Teil der Projektplanung. Es soll die Ziele und Anforderungen an die Bachelor Thesis \textit{CAVE Rendering Framework} wiedergeben. Es soll das Vorgehen und die Form der Arbeit regeln und nach Projektabschluss das Messen der Ziele erlauben.

\paragraph{}
Einleitend werden die verschiedenen bereits bestehenden Ressourcen, die in der Bachelor Thesis zum Einsatz kommen, n\"aher erkl\"art. Somit sollte es auch f\"ur Informatiker ohne Hintergrundwissen in \textit{Virtual Reality} (VR) m\"oglich sein, das Pflichtenheft zu verstehen. Fachspezifische Begriffe und Abk\"urzungen werden im Glossar oder in Fussnoten erl\"autert.