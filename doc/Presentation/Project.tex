\section{Project description}

% Initial Setup
\subsection{Initial setup}
\begin{frame}\frametitle{Initial setup}
	\begin{itemize}
		\item<1-> CAVE
		\begin{itemize}
			\item 3 walls
			\item 2 projectors per wall
		\end{itemize}
		\item<2-> Previous solutions
		\begin{itemize}
			\item Chromium
			\item VRML/X3D
		\end{itemize}
%		\item<3-> New solution
%		\begin{itemize}
%			\item Scene graph oriented
%			\item Equalizer based
%		\end{itemize}
	\end{itemize}
\end{frame}

% Task Description
\subsection{Task description}

\begin{frame}\frametitle{Task description}
	\begin{quote}
		The goal of this bachelor thesis is the development of a CAVE Rendering Framework for C++/OpenGL applications. The framework should be built on common libraries for scene-graph based rendering such as OpenSceneGraph (www.oppensceengraph.org) or the OGRE Game Engine (www.ogre3d.org). For distributed rendering one of these libraries needs to be integrated with the Equalizer Library (www.equalizergraphics.com). The goal is to develop a CAVE Rendering Framework for simplified implementation of C++/OpenGL CAVE applications.
	\end{quote}
\end{frame}

% Goals
\subsection{Goals}
\begin{frame}\frametitle{Goals}
	\begin{itemize}
		\item<1-> Primary goals
		\begin{itemize}
			\item Integrate a scene graph in Equalizer
			\item Stereoscopic output
			\item OpenGL based solution
			\item Performant scene graph rendering
			\item Extensibility
			\item Configurability
		\end{itemize}
		\item<2-> Secondary goals
		\begin{itemize}
			\item Volume rendering
			\item Haptics
			\item Shader
			\item Physics
			\item Tracking
		\end{itemize}
	\end{itemize}
\end{frame}


