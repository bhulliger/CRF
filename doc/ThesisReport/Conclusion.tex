% ==========================================
% Thesis Report - Conclusion ( all )
% ==========================================

\chapter{Conclusion}
After a tough start and some disappointing first test results, we could determine the key points and the difficulties of our project. A disappointing insight was the fact, that \gls{osg} provides a lot of features but it was pretty hard to find tutorials which were actual and of good quality. The API documentation, too, is rather poorly documentet. A lot of function descriptons are missing or detailed enough. Furthermore, we tried to find some advanced open source \gls{osg} applications to test our framework. Unfortunately, we could find nothing but some old and unspectacular examples.

Finally, with the hard-earned knowledge and our final test environment, we were able to create a fully working and easy-to-use framework for the rendering of \gls{osg} scene graphs with Equalizer. This framework is compatible with most of the possible and useful Equalizer topologies (not everything could be tested), renders the \gls{osg} scene graph with high performance and provides basic solutions for event handling and dynamic scene graph manipulation. Furthermore, the \gls{crf} is able to display a lot of useful statistics and should be easily extendible with new features like tracking, haptics and others.
