% ==========================================
% Thesis Report - Project Management ( waltj3 )
% ==========================================
\chapter{Project Management}

\section{Strategy}
When we first discussed our process strategy we decided to use the sequential waterfall model, because all team members already worked with this kind of project management. But further discussions with our team assistant and thesis expert lead us to use a more iterative approach. After all, we used a combination of the classic waterfall model and the iterative approach.
Because of the lack of knowledge and the complexity of our project, several maxims had to be tried out. Hence, a lot of small prototypes were built and tested to figure out what works and what does not. All these prototypes formed an iterative process model with the components \emph{Analysis and Design}, \emph{Implementation} and \emph{Testing}. 

\section{Milestones}
Please consider the functional specification document (\emph{Pflichtenheft}, written in German) attached in the appendix. This section refers to the initially defined milestones in the mentioned document. 
For each milestone the objectives are listed and if they were achieved. Moreover, it is listed until when they were initially planed and when they were achieved. Delays are marked red while early achievements are marked green. 

\subsection{M1: Functional Specification created}

\begin{table}[H]
	\centering
	\begin{tabular}{|p{0.5\textwidth}|p{0.12\textwidth}|p{0.12\textwidth}|p{0.12\textwidth}|}
		\hline 
		\bfseries Objective & \bfseries Achieved & \bfseries Date expected  & \bfseries Date realised \\ 
		\hline
		\hline 
		Functional specification completed  	& \tick & 18.03	&	18.03	 \\
		\hline 
	\end{tabular}
	\caption{Objectives of milestones 1}
\end{table}

\subsection{M2: Prototype finished}
This milestone changed in his manner. Instead of developing one prototype, we developed several to figure out the best way to go. So, this milestone was cycled multiple times. Thus, we needed some extra time to achieve this task.

\begin{table}[H]
	\centering
	\begin{tabular}{|p{0.5\textwidth}|p{0.12\textwidth}|p{0.12\textwidth}|p{0.12\textwidth}|}
		\hline 
		\bfseries Objective & \bfseries Achieved & \bfseries Date expected  & \bfseries Date realised \\ 
		\hline
		\hline 
		Final test cases defined  	& \tick & 18.03	&	\textcolor{red}{29.03}	 \\
		\hline 
		Project set up defined  	& \tick & 18.03	&	18.03	 \\
		\hline 
		Techniques to use defined  	& \tick & 18.03	&	18.03	 \\
		\hline 
	\end{tabular}
	\caption{Updated objectives of milestone two}
\end{table}

\subsection{M3: Developing finished}

\subsubsection{M3.1: Design Concept realised}
\label{sec:M31}
Based on the gathered experience of the prototypes, we were able to define a final software design for our framework.

\begin{table}[H]
	\centering
	\begin{tabular}{|p{0.5\textwidth}|p{0.12\textwidth}|p{0.12\textwidth}|p{0.12\textwidth}|}
		\hline 
		\bfseries Objective & \bfseries Achieved & \bfseries Date expected  & \bfseries Date realised \\ 
		\hline
		\hline 
		Class diagram created			& \tick				& 29.05		& \textcolor{green}{24.04} \\
		\hline 
		Sequence diagram  			& \textcolor{yellow}{partially}\footnotemark[1]	& 29.05		& 29.05 \\
		\hline 
		Techniques to use are well defined  	& \tick				& 18.03		& 18.03	\\
		\hline 
	\end{tabular}
	\caption{Updated objectives of milestone 3.1}
\end{table}
\footnotetext[1]{Too much of the processes use Equalizer. It would have been far to complex to depict sequence diagrams of good quality.}


\subsubsection{M3.2: Implementation finished}
\label{sec:M32}

The user guide has been split up in a user manual on how to  set up the BFH-TI and a programming guide about our CAVE Rendering Framework. 

\begin{table}[H]
	\centering
	\begin{tabular}{|p{0.5\textwidth}|p{0.12\textwidth}|p{0.12\textwidth}|p{0.12\textwidth}|}
		\hline 
		\bfseries Objective & \bfseries Achieved & \bfseries Date expected  & \bfseries Date realised \\ 
		\hline
		\hline 
		Technical documentation completed	& \tick 				& 29.05	& \textcolor{green}{28.05} \\
		\hline 
		User guide completed 			& \tick 				& 04.06	& \textcolor{red}{06.06} \\
		\hline 
		Programming guide completed 		& \tick					& 04.06	& \textcolor{red}{09.06} \\
		\hline 
		Primary use cases implemented		& \tick 				& 29.03	& \textcolor{red}{24.05} \\
		\hline 
		Primary requirements achieved		& \textcolor{yellow}{partially}\footnotemark[2] & 29.03	& \textcolor{red}{24.05} \\
		\hline 
		Runnable demo application		& \tick 				& 29.03	& \textcolor{red}{18.05} \\
		\hline 
	\end{tabular}
	\caption{Updated milestones 3.2 criteria}
\end{table}
\footnotetext[2]{The rendering performance on multipipe configurations is insufficient}

\subsubsection{M4: Tests finished}
\label{sec:M4TestsFinished}
\begin{table}[H]
	\centering
	\begin{tabular}{|p{0.5\textwidth}|p{0.12\textwidth}|p{0.12\textwidth}|p{0.12\textwidth}|}
		\hline 
		\bfseries Objective & \bfseries Achieved & \bfseries Date expected  & \bfseries Date realised \\ 
		\hline
		\hline 
		Test reports available 				& \tick & 12.06	& \textcolor{green}{09.06}	\\
		\hline 
		Bugs fixed 					& \tick & 29.05	& 29.05 \\
		\hline 
		All demo and test applications run well  	& \tick & 12.06	& \textcolor{green}{10.06}	\\
		\hline 
	\end{tabular}
	\caption{Updated objectives of milestone 4}
\end{table}

\subsubsection{M5: Documentation/Project finished}
\label{sec:M5DocumentationProjectFinished}
\begin{table}[H]
	\centering
	\begin{tabular}{|p{0.5\textwidth}|p{0.12\textwidth}|p{0.12\textwidth}|p{0.12\textwidth}|}
		\hline 
		\bfseries Objective & \bfseries Achieved & \bfseries Date expected  & \bfseries Date realised \\ 
		\hline
		\hline 
		Milestones 1-4 achieved 			& \tick & 12.06 &	12.06 \\
		\hline 
		Technical documentation finished 		& \tick & 29.05 &	\textcolor{red}{10.06} \\
		\hline 
		User guide and programming guide finished  	& \tick & 12.06 &	\textcolor{green}{11.06} \\
		\hline 
	\end{tabular}
	\caption{Updated milestones 5 criteria}
\end{table}

\section{Appointments}
\label{sec:MeetingWithVanDerKleij}


\begin{table}[H]
	\centering
	\begin{tabular}{|b{0.2\textwidth}|b{0.75\textwidth}|}
		\hline
 		\multicolumn{2}{|l|}{\bfseries Kick-Off meeting with whole project group} \\
		\hline\hline
		\bfseries Date & 20.02.2009 \\
		\hline
		\bfseries Location & \gls{bfhti} \\
		\hline
		\bfseries Participants & - Prof. U. K\"unzler \\
		&- M. Luggen\\
		&- J. Walti\\
		&- B. Hulliger\\
		&- S. Broder \\
		\hline
		\bfseries Substance & - content of the functional specification document \\
		&- project environment \\
		&- requirements \\
		&- use cases\\
		&- contacts\\
		&- milestones and deadlines \\
		\hline
	\end{tabular}
\end{table}

\begin{table}[H]
	\centering
	\begin{tabular}{|b{0.2\textwidth}|b{0.75\textwidth}|}
		\hline
 		\multicolumn{2}{|l|}{\bfseries Review of functional specification document} \\
		\hline\hline
		\bfseries Date & 09.03.2009 \\
		\hline
		\bfseries Location & \gls{bfhti} \\
		\hline
		\bfseries Participants & - Prof. U. K\"unzler \\
		&- M. Luggen\\
		&- J. Walti\\
		&- B. Hulliger\\
		&- S. Broder \\
		\hline
		\bfseries Substance & We discussed our functional specification document, what adjustments we had to make.\\
		\hline
	\end{tabular}
\end{table}

\begin{table}[H]
	\centering
	\begin{tabular}{|b{0.2\textwidth}|b{0.75\textwidth}|}
		\hline
 		\multicolumn{2}{|l|}{\bfseries Meeting with expert H. van der Kleij} \\
		\hline\hline
		\bfseries Date & 20.03.2009 \\
		\hline
		\bfseries Location & SBB, Bollwerk Berne \\
		\hline
		\bfseries Participants & - H. van der Kleij \\
		&- J. Walti\\
		&- B. Hulliger\\
		&- S. Broder \\
		\hline
		\bfseries Substance & - project overview \\
		&- review of the functional specification document \\
		&- further procedure\\
		\hline
	\end{tabular}
\end{table}

\begin{table}[H]
	\centering
	\begin{tabular}{|b{0.2\textwidth}|b{0.75\textwidth}|}
		\hline
 		\multicolumn{2}{|l|}{\bfseries eqOSG Meeting at the Eurographics 2009 in Munich} \\
		\hline\hline
		\bfseries Date & 31.03.2009 \\
		\hline
		\bfseries Location & Eurographics, University of Munich \\
		\hline
		\bfseries Participants & - S. Eilemann \& Crew (Equalizer, Eyescale) \\
		&- M. Luggen, J. Walti, B. Hulliger, S. Broder (\gls{bfhti}) \\
		&- Thomas McGuire \& Team (University of Siegen) \\
		\hline
		\bfseries Substance & This meeting was very interesting because a group of students from the University of Siegen already developed a comparable solution in a similar project. We could exchange some technical facts and see what they have done so fare. S. Eilemann was particularly interested in a generic integration of \gls{osg} into Equalizer.\\
		\hline
	\end{tabular}
\end{table}

\begin{table}[H]
	\centering
	\begin{tabular}{|b{0.2\textwidth}|b{0.75\textwidth}|}
		\hline
 		\multicolumn{2}{|l|}{\bfseries Periodic meetings with supervisor M. Luggen} \\
		\hline\hline
		\bfseries Date & every two weeks \\
		\hline
		\bfseries Location & \gls{bfhti} \\
		\hline
		\bfseries Participants & - M. Luggen \\
		& - J. Walti \\
		& - B. Hulliger \\
		& - S. Broder \\
		\hline
		\bfseries Substance & These meetings were excellent for experience exchange with M. Luggen, who gave us regularly worthwile inputs. We discussed general project decisions during these meetings and showed Michael Luggen what we already achieved. \\
		\hline
	\end{tabular}
\end{table}