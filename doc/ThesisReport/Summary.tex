% ==========================================
% Thesis Report - Summary ( all )
% ==========================================

\chapter{Personal Summary}

The developing of a simple framework for parallel rendering was stated as main goal of our project work. In the first phase of our project, we all tried to get a closer look on the Equalizer library and parallel rendering in general. This effort was expected and we were all quite eager to learn about parallel rendering. Thereafter, we had to put quite a bit time in the hardware setup. There was the \gls{cave} clone to setup and some \gls{cave} maintenance and cleanup that needed to be done. This was more time consuming than we expected. So, the the whole project delayed because the testing of the framework could not be done in a real environment and some ideas for the framework ended in an impasse, which, again, did cost us a lot of time. Nevertheless, some of the early ideas survived all the tests and are now part of the final release.

The idea exchange with the team from the university of Siegen was valuable and helped us at the beginning to get the right direction regarding the integration of \gls{osg} into Equalizer. We did reuse some parts of their solution, improved it for our purposes and eliminated some errors. After the first running prototypes we searched a way to simplify the whole thing which was one of our main goals. This was quite difficult because an Equalizer application is not simple and we could deskill a lot, but not all of it. Especially the rendering of a dynamic scene graph is very difficult with Equalizer, because the interaction between the different rendering outputs, what also has to work over the network for example, needs a kind of specialised protocol. We could not find a way to simplify this in general.  At least all the mouse and keyboard inputs are committed to all the scene graphs and makes it now possible to change the scene graphs synchronously.

The development of the \gls{crf} was accompanied by configuring Equalizer. This, as well, did cost us a lot of time since our setup was pretty exceptional. Therefore, a lot of effort went into internet research and mailing lists. Initially, we tried address multiple graphics cards in Windows Vista what did turn out to be impossible at the moment. Thereafter, we had trouble setting up a network configuration of Equalizer. In fact, it reavealed to be a general problem when using 32-bit and 64-bit \gls{cpu} architectures combined. After all, Equalizer seems to work out of the box for standard setups, but not so much for rather exotic setups.

To sum up, we all learned a lot about the difficulties of parallel rendering and framework development. The most important part of such a project is testing. To develop in complex surroundings like these different parallel rendering topologies, it is above all necessary to test under real conditions. If we could do such a project once again, we would first focus on the setup of the \gls{cave} environment to ensure the availability of a working setup. This would have boost the development of the framework tremendously and thus, our result would be much more enhanced. The approach to do the \gls{cave} setup and the development concurrently was kind of intuitiv, because of our limited work and time resources. Eventually, it might have been the better choice to only focus on the setups first.

