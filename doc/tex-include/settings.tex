%%%%%%%%%%%%%%%%%%%%%%%%%%%%%%%%%%%%%%%%%%%%%%%%%%%%%
% settings                                          %
%                                                   %
% einbinden mit: %%%%%%%%%%%%%%%%%%%%%%%%%%%%%%%%%%%%%%%%%%%%%%%%%%%%%
% settings                                          %
%                                                   %
% einbinden mit: %%%%%%%%%%%%%%%%%%%%%%%%%%%%%%%%%%%%%%%%%%%%%%%%%%%%%
% settings                                          %
%                                                   %
% einbinden mit: %%%%%%%%%%%%%%%%%%%%%%%%%%%%%%%%%%%%%%%%%%%%%%%%%%%%%
% settings                                          %
%                                                   %
% einbinden mit: \input{settings}                   %
%%%%%%%%%%%%%%%%%%%%%%%%%%%%%%%%%%%%%%%%%%%%%%%%%%%%%

\documentclass[a4paper, 11pt, oneside, openany,abstracton]{scrreprt} % koma-script layout

% --------------------------------------
% alle packages die wir brauchen:
% --------------------------------------

% Schriften, Sprache, Text:
\usepackage[T1]{fontenc}
\usepackage[latin1]{inputenc}
%\usepackage[german]{babel}
%\usepackage{../tex-include/german} %sonst gehen die Anfhrungzeichen nicht!
%\usepackage{../tex-include/ngerman}
%\usepackage[english]{babel}
%\usepackage{english}

\usepackage[T1]{fontenc}
\usepackage[scaled]{../tex-include/uarial}
\renewcommand*\familydefault{\sfdefault} %% Only if the base font of the document is to be sans serif

\usepackage{hyperref}
\usepackage{../tex-include/glossaries}
\makeglossaries

\usepackage{pifont}
\newcommand{\tick}{\quad\quad\color{green}{\ding{52}}}
\newcommand{\partialTick}{\quad\quad\color{green}{(\ding{52})}}
\newcommand{\cross}{\quad\quad\color{red}{\ding{54}}}

\usepackage{pdfpages}


% wegen deutschen Umlauten
%\usepackage[ansinew]{inputenc}
% alles in Farbe:
\usepackage{color}
\usepackage{listings}
\lstset{ %
language=C,                		% choose the language of the code
basicstyle=\footnotesize,       % the size of the fonts that are used for the code
frame=single,			% adds a frame around the code
captionpos=b,			% sets the caption-position to bottom
breaklines=true,		% sets automatic line breaking
breakatwhitespace=false,	% sets if automatic breaks should only happen at whitespace
tabsize=2,
escapeinside={\%*}{*)}          % if you want to add a comment within your code
}

% Tabellen
\usepackage{tabularx}
\usepackage{../tex-include/multirow}
\usepackage{multicol}
\usepackage{longtable}
\usepackage{hhline}
\usepackage{booktabs}
% Boxen
\usepackage{../tex-include/pbox} 
% Aufzaehlungen
\usepackage{../tex-include/shortlst}
% Equation array mit mehrfacher Ausrichtung
\usepackage{array}
% verbatim
\usepackage{fancyvrb}
% Grafik:
\usepackage{epsfig}
\usepackage{../tex-include/texdraw}
\usepackage{graphicx}
\usepackage{flafter}
\usepackage[bf]{caption2}  % Wort Abbildung bold
\usepackage{float} % grafik genau hier einbinden mit [H]
% AMS-Math:
\usepackage{amsmath}
\usepackage{amstext}
\usepackage{amstext}
\usepackage{amsfonts}
\usepackage{amsxtra}
\usepackage{amssymb}
%\usepackage{../tex-include/stmaryrd} % blitz
\usepackage{../tex-include/wasysym}  % varangle
% double Stroke Fonts
\usepackage{../tex-include/bbm}
% Index, Headers
\usepackage{makeidx}
\usepackage{fancyhdr}
\usepackage{lscape}
\usepackage{enumerate}
% Schrift fr PDF wegen [T1]
\usepackage{ae,aecompl}



%\usepackage{hyperref}
% \usepackage[
%   colorlinks,
%   linkcolor=blue,
%   urlcolor=red%
% ]{hyperref}

\usepackage{color}
\definecolor{darkblue}{rgb}{0,0.1,0.5}
\hypersetup{colorlinks,
            linkcolor=darkblue,
            anchorcolor=darkblue,
            citecolor=darkblue}

%\definecolor{darkblue}{rgb}{0,0.1,0.5}
\hypersetup{pdftex=true, colorlinks=false, breaklinks=true, linkcolor=darkblue, menucolor=darkblue, pagecolor=darkblue, urlcolor=darkblue}

% Zum Programmieren
\usepackage{ifthen}
\usepackage{calc}
%struktogramme erstellen


%-------------------------
% Bibliography, index
%-------------------------
% fuer Zitate
\usepackage[square]{natbib}

% Festlegung Art der Zitierung - Havardmethode: Abkuerzung Autor + Jahr
%\bibliographystyle{IEEEtranSA}
\bibliographystyle{plain}

% Stichwortverzeichnis erstellen
%\makeindex

% --------------------------------------
% Seitenraender neu einstellen:
% --------------------------------------

% einstellungen r�der fuer caption2
\setcaptionwidth{0.8\textwidth}


%einstellungen fr list umgebung
\setlength{\topsep}{0pt}
\setlength{\itemsep}{0.2pt}
\setlength{\parsep}{0pt}

%einstellungen fr seitenlayout
\setlength{\marginparwidth}{0pt}
\setlength{\textwidth}{440pt}
\setlength{\hoffset}{-20pt}
\setlength{\voffset}{-20pt}
\setlength{\evensidemargin}{0pt}
\setlength{\topmargin}{0pt}
\setlength{\headheight}{14pt}
\setlength{\headsep}{18pt}
\setlength{\textheight}{670pt}
\setlength{\marginparsep}{0pt}
\setlength{\marginparpush}{5pt}
\setlength{\footskip}{27pt}



% --------------------------------------
% Einstellungen fr das Koma Skript
% --------------------------------------

% weniger leerraum ber dem chapter titel
\renewcommand*{\chapterheadstartvskip}{\vspace*{-1cm}}



                   %
%%%%%%%%%%%%%%%%%%%%%%%%%%%%%%%%%%%%%%%%%%%%%%%%%%%%%

\documentclass[a4paper, 11pt, oneside, openany,abstracton]{scrreprt} % koma-script layout

% --------------------------------------
% alle packages die wir brauchen:
% --------------------------------------

% Schriften, Sprache, Text:
\usepackage[T1]{fontenc}
\usepackage[latin1]{inputenc}
%\usepackage[german]{babel}
%\usepackage{../tex-include/german} %sonst gehen die Anfhrungzeichen nicht!
%\usepackage{../tex-include/ngerman}
%\usepackage[english]{babel}
%\usepackage{english}

\usepackage[T1]{fontenc}
\usepackage[scaled]{../tex-include/uarial}
\renewcommand*\familydefault{\sfdefault} %% Only if the base font of the document is to be sans serif

\usepackage{hyperref}
\usepackage{../tex-include/glossaries}
\makeglossaries

\usepackage{pifont}
\newcommand{\tick}{\quad\quad\color{green}{\ding{52}}}
\newcommand{\partialTick}{\quad\quad\color{green}{(\ding{52})}}
\newcommand{\cross}{\quad\quad\color{red}{\ding{54}}}

\usepackage{pdfpages}


% wegen deutschen Umlauten
%\usepackage[ansinew]{inputenc}
% alles in Farbe:
\usepackage{color}
\usepackage{listings}
\lstset{ %
language=C,                		% choose the language of the code
basicstyle=\footnotesize,       % the size of the fonts that are used for the code
frame=single,			% adds a frame around the code
captionpos=b,			% sets the caption-position to bottom
breaklines=true,		% sets automatic line breaking
breakatwhitespace=false,	% sets if automatic breaks should only happen at whitespace
tabsize=2,
escapeinside={\%*}{*)}          % if you want to add a comment within your code
}

% Tabellen
\usepackage{tabularx}
\usepackage{../tex-include/multirow}
\usepackage{multicol}
\usepackage{longtable}
\usepackage{hhline}
\usepackage{booktabs}
% Boxen
\usepackage{../tex-include/pbox} 
% Aufzaehlungen
\usepackage{../tex-include/shortlst}
% Equation array mit mehrfacher Ausrichtung
\usepackage{array}
% verbatim
\usepackage{fancyvrb}
% Grafik:
\usepackage{epsfig}
\usepackage{../tex-include/texdraw}
\usepackage{graphicx}
\usepackage{flafter}
\usepackage[bf]{caption2}  % Wort Abbildung bold
\usepackage{float} % grafik genau hier einbinden mit [H]
% AMS-Math:
\usepackage{amsmath}
\usepackage{amstext}
\usepackage{amstext}
\usepackage{amsfonts}
\usepackage{amsxtra}
\usepackage{amssymb}
%\usepackage{../tex-include/stmaryrd} % blitz
\usepackage{../tex-include/wasysym}  % varangle
% double Stroke Fonts
\usepackage{../tex-include/bbm}
% Index, Headers
\usepackage{makeidx}
\usepackage{fancyhdr}
\usepackage{lscape}
\usepackage{enumerate}
% Schrift fr PDF wegen [T1]
\usepackage{ae,aecompl}



%\usepackage{hyperref}
% \usepackage[
%   colorlinks,
%   linkcolor=blue,
%   urlcolor=red%
% ]{hyperref}

\usepackage{color}
\definecolor{darkblue}{rgb}{0,0.1,0.5}
\hypersetup{colorlinks,
            linkcolor=darkblue,
            anchorcolor=darkblue,
            citecolor=darkblue}

%\definecolor{darkblue}{rgb}{0,0.1,0.5}
\hypersetup{pdftex=true, colorlinks=false, breaklinks=true, linkcolor=darkblue, menucolor=darkblue, pagecolor=darkblue, urlcolor=darkblue}

% Zum Programmieren
\usepackage{ifthen}
\usepackage{calc}
%struktogramme erstellen


%-------------------------
% Bibliography, index
%-------------------------
% fuer Zitate
\usepackage[square]{natbib}

% Festlegung Art der Zitierung - Havardmethode: Abkuerzung Autor + Jahr
%\bibliographystyle{IEEEtranSA}
\bibliographystyle{plain}

% Stichwortverzeichnis erstellen
%\makeindex

% --------------------------------------
% Seitenraender neu einstellen:
% --------------------------------------

% einstellungen r�der fuer caption2
\setcaptionwidth{0.8\textwidth}


%einstellungen fr list umgebung
\setlength{\topsep}{0pt}
\setlength{\itemsep}{0.2pt}
\setlength{\parsep}{0pt}

%einstellungen fr seitenlayout
\setlength{\marginparwidth}{0pt}
\setlength{\textwidth}{440pt}
\setlength{\hoffset}{-20pt}
\setlength{\voffset}{-20pt}
\setlength{\evensidemargin}{0pt}
\setlength{\topmargin}{0pt}
\setlength{\headheight}{14pt}
\setlength{\headsep}{18pt}
\setlength{\textheight}{670pt}
\setlength{\marginparsep}{0pt}
\setlength{\marginparpush}{5pt}
\setlength{\footskip}{27pt}



% --------------------------------------
% Einstellungen fr das Koma Skript
% --------------------------------------

% weniger leerraum ber dem chapter titel
\renewcommand*{\chapterheadstartvskip}{\vspace*{-1cm}}



                   %
%%%%%%%%%%%%%%%%%%%%%%%%%%%%%%%%%%%%%%%%%%%%%%%%%%%%%

\documentclass[a4paper, 11pt, oneside, openany,abstracton]{scrreprt} % koma-script layout

% --------------------------------------
% alle packages die wir brauchen:
% --------------------------------------

% Schriften, Sprache, Text:
\usepackage[T1]{fontenc}
\usepackage[latin1]{inputenc}
%\usepackage[german]{babel}
%\usepackage{../tex-include/german} %sonst gehen die Anfhrungzeichen nicht!
%\usepackage{../tex-include/ngerman}
%\usepackage[english]{babel}
%\usepackage{english}

\usepackage[T1]{fontenc}
\usepackage[scaled]{../tex-include/uarial}
\renewcommand*\familydefault{\sfdefault} %% Only if the base font of the document is to be sans serif

\usepackage{hyperref}
\usepackage{../tex-include/glossaries}
\makeglossaries

\usepackage{pifont}
\newcommand{\tick}{\quad\quad\color{green}{\ding{52}}}
\newcommand{\partialTick}{\quad\quad\color{green}{(\ding{52})}}
\newcommand{\cross}{\quad\quad\color{red}{\ding{54}}}

\usepackage{pdfpages}


% wegen deutschen Umlauten
%\usepackage[ansinew]{inputenc}
% alles in Farbe:
\usepackage{color}
\usepackage{listings}
\lstset{ %
language=C,                		% choose the language of the code
basicstyle=\footnotesize,       % the size of the fonts that are used for the code
frame=single,			% adds a frame around the code
captionpos=b,			% sets the caption-position to bottom
breaklines=true,		% sets automatic line breaking
breakatwhitespace=false,	% sets if automatic breaks should only happen at whitespace
tabsize=2,
escapeinside={\%*}{*)}          % if you want to add a comment within your code
}

% Tabellen
\usepackage{tabularx}
\usepackage{../tex-include/multirow}
\usepackage{multicol}
\usepackage{longtable}
\usepackage{hhline}
\usepackage{booktabs}
% Boxen
\usepackage{../tex-include/pbox} 
% Aufzaehlungen
\usepackage{../tex-include/shortlst}
% Equation array mit mehrfacher Ausrichtung
\usepackage{array}
% verbatim
\usepackage{fancyvrb}
% Grafik:
\usepackage{epsfig}
\usepackage{../tex-include/texdraw}
\usepackage{graphicx}
\usepackage{flafter}
\usepackage[bf]{caption2}  % Wort Abbildung bold
\usepackage{float} % grafik genau hier einbinden mit [H]
% AMS-Math:
\usepackage{amsmath}
\usepackage{amstext}
\usepackage{amstext}
\usepackage{amsfonts}
\usepackage{amsxtra}
\usepackage{amssymb}
%\usepackage{../tex-include/stmaryrd} % blitz
\usepackage{../tex-include/wasysym}  % varangle
% double Stroke Fonts
\usepackage{../tex-include/bbm}
% Index, Headers
\usepackage{makeidx}
\usepackage{fancyhdr}
\usepackage{lscape}
\usepackage{enumerate}
% Schrift fr PDF wegen [T1]
\usepackage{ae,aecompl}



%\usepackage{hyperref}
% \usepackage[
%   colorlinks,
%   linkcolor=blue,
%   urlcolor=red%
% ]{hyperref}

\usepackage{color}
\definecolor{darkblue}{rgb}{0,0.1,0.5}
\hypersetup{colorlinks,
            linkcolor=darkblue,
            anchorcolor=darkblue,
            citecolor=darkblue}

%\definecolor{darkblue}{rgb}{0,0.1,0.5}
\hypersetup{pdftex=true, colorlinks=false, breaklinks=true, linkcolor=darkblue, menucolor=darkblue, pagecolor=darkblue, urlcolor=darkblue}

% Zum Programmieren
\usepackage{ifthen}
\usepackage{calc}
%struktogramme erstellen


%-------------------------
% Bibliography, index
%-------------------------
% fuer Zitate
\usepackage[square]{natbib}

% Festlegung Art der Zitierung - Havardmethode: Abkuerzung Autor + Jahr
%\bibliographystyle{IEEEtranSA}
\bibliographystyle{plain}

% Stichwortverzeichnis erstellen
%\makeindex

% --------------------------------------
% Seitenraender neu einstellen:
% --------------------------------------

% einstellungen r�der fuer caption2
\setcaptionwidth{0.8\textwidth}


%einstellungen fr list umgebung
\setlength{\topsep}{0pt}
\setlength{\itemsep}{0.2pt}
\setlength{\parsep}{0pt}

%einstellungen fr seitenlayout
\setlength{\marginparwidth}{0pt}
\setlength{\textwidth}{440pt}
\setlength{\hoffset}{-20pt}
\setlength{\voffset}{-20pt}
\setlength{\evensidemargin}{0pt}
\setlength{\topmargin}{0pt}
\setlength{\headheight}{14pt}
\setlength{\headsep}{18pt}
\setlength{\textheight}{670pt}
\setlength{\marginparsep}{0pt}
\setlength{\marginparpush}{5pt}
\setlength{\footskip}{27pt}



% --------------------------------------
% Einstellungen fr das Koma Skript
% --------------------------------------

% weniger leerraum ber dem chapter titel
\renewcommand*{\chapterheadstartvskip}{\vspace*{-1cm}}



                   %
%%%%%%%%%%%%%%%%%%%%%%%%%%%%%%%%%%%%%%%%%%%%%%%%%%%%%

\documentclass[a4paper, 11pt, oneside, openany,abstracton]{scrreprt} % koma-script layout

% --------------------------------------
% alle packages die wir brauchen:
% --------------------------------------

% Schriften, Sprache, Text:
\usepackage[T1]{fontenc}
\usepackage[latin1]{inputenc}
%\usepackage[german]{babel}
%\usepackage{../tex-include/german} %sonst gehen die Anfhrungzeichen nicht!
%\usepackage{../tex-include/ngerman}
%\usepackage[english]{babel}
%\usepackage{english}

\usepackage[T1]{fontenc}
\usepackage[scaled]{../tex-include/uarial}
\renewcommand*\familydefault{\sfdefault} %% Only if the base font of the document is to be sans serif

\usepackage{hyperref}
\usepackage{../tex-include/glossaries}
\makeglossaries

\usepackage{pifont}
\newcommand{\tick}{\quad\quad\color{green}{\ding{52}}}
\newcommand{\partialTick}{\quad\quad\color{green}{(\ding{52})}}
\newcommand{\cross}{\quad\quad\color{red}{\ding{54}}}

\usepackage{pdfpages}


% wegen deutschen Umlauten
%\usepackage[ansinew]{inputenc}
% alles in Farbe:
\usepackage{color}
\usepackage{listings}
\lstset{ %
language=C,                		% choose the language of the code
basicstyle=\footnotesize,       % the size of the fonts that are used for the code
frame=single,			% adds a frame around the code
captionpos=b,			% sets the caption-position to bottom
breaklines=true,		% sets automatic line breaking
breakatwhitespace=false,	% sets if automatic breaks should only happen at whitespace
tabsize=2,
escapeinside={\%*}{*)}          % if you want to add a comment within your code
}

% Tabellen
\usepackage{tabularx}
\usepackage{../tex-include/multirow}
\usepackage{multicol}
\usepackage{longtable}
\usepackage{hhline}
\usepackage{booktabs}
% Boxen
\usepackage{../tex-include/pbox} 
% Aufzaehlungen
\usepackage{../tex-include/shortlst}
% Equation array mit mehrfacher Ausrichtung
\usepackage{array}
% verbatim
\usepackage{fancyvrb}
% Grafik:
\usepackage{epsfig}
\usepackage{../tex-include/texdraw}
\usepackage{graphicx}
\usepackage{flafter}
\usepackage[bf]{caption2}  % Wort Abbildung bold
\usepackage{float} % grafik genau hier einbinden mit [H]
% AMS-Math:
\usepackage{amsmath}
\usepackage{amstext}
\usepackage{amstext}
\usepackage{amsfonts}
\usepackage{amsxtra}
\usepackage{amssymb}
%\usepackage{../tex-include/stmaryrd} % blitz
\usepackage{../tex-include/wasysym}  % varangle
% double Stroke Fonts
\usepackage{../tex-include/bbm}
% Index, Headers
\usepackage{makeidx}
\usepackage{fancyhdr}
\usepackage{lscape}
\usepackage{enumerate}
% Schrift fr PDF wegen [T1]
\usepackage{ae,aecompl}



%\usepackage{hyperref}
% \usepackage[
%   colorlinks,
%   linkcolor=blue,
%   urlcolor=red%
% ]{hyperref}

\usepackage{color}
\definecolor{darkblue}{rgb}{0,0.1,0.5}
\hypersetup{colorlinks,
            linkcolor=darkblue,
            anchorcolor=darkblue,
            citecolor=darkblue}

%\definecolor{darkblue}{rgb}{0,0.1,0.5}
\hypersetup{pdftex=true, colorlinks=false, breaklinks=true, linkcolor=darkblue, menucolor=darkblue, pagecolor=darkblue, urlcolor=darkblue}

% Zum Programmieren
\usepackage{ifthen}
\usepackage{calc}
%struktogramme erstellen


%-------------------------
% Bibliography, index
%-------------------------
% fuer Zitate
\usepackage[square]{natbib}

% Festlegung Art der Zitierung - Havardmethode: Abkuerzung Autor + Jahr
%\bibliographystyle{IEEEtranSA}
\bibliographystyle{plain}

% Stichwortverzeichnis erstellen
%\makeindex

% --------------------------------------
% Seitenraender neu einstellen:
% --------------------------------------

% einstellungen r�der fuer caption2
\setcaptionwidth{0.8\textwidth}


%einstellungen fr list umgebung
\setlength{\topsep}{0pt}
\setlength{\itemsep}{0.2pt}
\setlength{\parsep}{0pt}

%einstellungen fr seitenlayout
\setlength{\marginparwidth}{0pt}
\setlength{\textwidth}{440pt}
\setlength{\hoffset}{-20pt}
\setlength{\voffset}{-20pt}
\setlength{\evensidemargin}{0pt}
\setlength{\topmargin}{0pt}
\setlength{\headheight}{14pt}
\setlength{\headsep}{18pt}
\setlength{\textheight}{670pt}
\setlength{\marginparsep}{0pt}
\setlength{\marginparpush}{5pt}
\setlength{\footskip}{27pt}



% --------------------------------------
% Einstellungen fr das Koma Skript
% --------------------------------------

% weniger leerraum ber dem chapter titel
\renewcommand*{\chapterheadstartvskip}{\vspace*{-1cm}}



